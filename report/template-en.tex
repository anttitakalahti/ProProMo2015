% --- Template for thesis / report with tktltiki2 class ---
% 
% last updated 2013/02/15 for tkltiki2 v1.02

\documentclass[english]{tktltiki2}

% tktltiki2 automatically loads babel, so you can simply
% give the language parameter (e.g. finnish, swedish, english, british) as
% a parameter for the class: \documentclass[finnish]{tktltiki2}.
% The information on title and abstract is generated automatically depending on
% the language, see below if you need to change any of these manually.
% 
% Class options:
% - grading                 -- Print labels for grading information on the front page.
% - disablelastpagecounter  -- Disables the automatic generation of page number information
%                              in the abstract. See also \numberofpagesinformation{} command below.
%
% The class also respects the following options of article class:
%   10pt, 11pt, 12pt, final, draft, oneside, twoside,
%   openright, openany, onecolumn, twocolumn, leqno, fleqn
%
% The default font size is 11pt. The paper size used is A4, other sizes are not supported.
%
% rubber: module pdftex

% --- General packages ---

\usepackage[utf8]{inputenc}
\usepackage[T1]{fontenc}
\usepackage{lmodern}
\usepackage{microtype}
\usepackage{amsfonts,amsmath,amssymb,amsthm,booktabs,color,enumitem,graphicx}
\usepackage[pdftex,hidelinks]{hyperref}

% Automatically set the PDF metadata fields
\makeatletter
\AtBeginDocument{\hypersetup{pdftitle = {\@title}, pdfauthor = {\@author}}}
\makeatother

% --- Language-related settings ---
%
% these should be modified according to your language

% babelbib for non-english bibliography using bibtex
\usepackage[fixlanguage]{babelbib}

% add bibliography to the table of contents
\usepackage[nottoc]{tocbibind}

% --- Theorem environment definitions ---

\newtheorem{thm}{Theorem}
\newtheorem{lem}[thm]{Lemma}
\newtheorem{cor}[thm]{Corollary}

\theoremstyle{definition}
\newtheorem{definition}[thm]{Definition}

\theoremstyle{remark}
\newtheorem*{remark}{Remark}


% --- tktltiki2 options ---
%
% The following commands define the information used to generate title and
% abstract pages. The following entries should be always specified:

\title{Project in Probabilistic Models}
\author{Antti Takalahti}
\date{\today}
\level{Seminar report}
\abstract{Abstract.}

% The following can be used to specify keywords and classification of the paper:

\keywords{keyword 1, keyword 2, keyword 3}

% classification according to ACM Computing Classification System (http://www.acm.org/about/class/)
% This is probably mostly relevant for computer scientists
% uncomment the following; contents of \classification will be printed under the abstract with a title
% "ACM Computing Classification System (CCS):"
% \classification{}

% If the automatic page number counting is not working as desired in your case,
% uncomment the following to manually set the number of pages displayed in the abstract page:
%
% \numberofpagesinformation{16 pages + 10 appendix pages}
%
% If you are not a computer scientist, you will want to uncomment the following by hand and specify
% your department, faculty and subject by hand:
%
% \faculty{Faculty of Science}
% \department{Department of Computer Science}
% \subject{Computer Science}
%
% If you are not from the University of Helsinki, then you will most likely want to set these also:
%
% \university{University of Helsinki}
% \universitylong{HELSINGIN YLIOPISTO --- HELSINGFORS UNIVERSITET --- UNIVERSITY OF HELSINKI} % displayed on the top of the abstract page
% \city{Helsinki}
%


\begin{document}

% --- Front matter ---

\frontmatter      % roman page numbering for front matter

\maketitle        % title page
% \makeabstract     % abstract page

\tableofcontents  % table of contents

% --- Main matter ---

\mainmatter       % clear page, start arabic page numbering

\section{Introduction}

This is a report for the Project in Probabilistic Models (582637) course. 

The task was to write a predictor that produces a probability distribution for 1000 rows of 303 column data. The source of the data is unknown and values are from 0 to 99 where 0 indicates that no measurement was made.

The task was split into three rounds, and a sample data was provided for each round. Total score was calculated by taking the natural logarithm for each probability per correct value and these were summed.

Initial algorithm was provided by Johannes Verwijnen, who was the teaching assistant for the course.


\section{First round}

I created a modular system where I could evaluate many different predictors to get an idea about their performance. I ended up using Java as a programming language as the example was written using Java. I added Apache Maven and set up Docker to have Maven installed. I had not used Docker before so this took some time, but I feel that it was time well spent. I feel that I got the system set up pretty nicely and it helped with implementing and evaluating different ideas about possible predictions.

I took a look peak into the data and noticed that the initial guess to predict previously seen value and its neighbour was silly and I managed to double the performance by removing that aspect. In the end I used a zero predictor for the first round with a score of -284 229.

\section{Second round}

First I converted all non zero values to x and searched for patterns, but sadly there weren?t any. I got 35 894 different patterns with two patterns having 5843 and 4189 rows and the rest had under 1000 rows each. The sample data had 67 785 rows total.

I then calculated the zero probabilities for each column and noticed that positions 16 (0.2718) and 50 (0.2636) look interesting. I started to predict these columns with zero probability < 0.5 and got better results. I ended up calculating smoothed probabilities for each value in each position and got a score of -89 150.

\section{Third round}

I made a matrix on how well a non zero value in each position predicts a non zero value in the following positions.

I created a class called ?Peak? to group values that are close together. 

Finally I noticed that peak values grew as time went by.

In the end I ended up using the same predictor from the second round.

\section{Thoughts}

I learned some real world skills.

I got some laughs from an old flame war on TDD and sudoku.

Predicting exact values was hard.

% --- References ---
%
% bibtex is used to generate the bibliography. The babplain style
% will generate numeric references (e.g. [1]) appropriate for theoretical
% computer science. If you need alphanumeric references (e.g [Tur90]), use
%
% \bibliographystyle{babalpha-lf}
%
% instead.

\bibliographystyle{babplain-lf}
\bibliography{references-en}


% --- Appendices ---

% uncomment the following

% \newpage
% \appendix
% 
% \section{Example appendix}

\end{document}
